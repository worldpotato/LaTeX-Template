\chapter{Zu meiner Person}
Für die Ausbildung zum IT-Systemelektroniker habe ich mich entschieden, da ich mich damals für Computer und elektronische Systeme interessiert habe. Diese Ausbildung hat meine Interessen damals sehr gut widergespiegelt und so konnte ich gelerntes auch in meinem Hobby, dem Modellsport, anwenden.

Gegen Ende der Ausbildung habe ich festgestellt, dass mir das vermittelte Wissen nicht ausreicht und so wollte ich meine Allgemeine Hochschulreife auf der Berufsoberschule nachholen um die Möglichkeit auf ein Studium zu haben.
Im Anschluss an meine bestandenen Prüfungen an der Berufsoberschule wollte ich allerdings etwas Berufspraxis sammeln.
Die InoNet Computer GmbH hat mir ein gutes Umfeld gegeben um meine Interessen um Computer, Software und Elektrotechnik im Beruflichen Umfeld zu vertiefen und anzuwenden.

Nach etwa zwei Jahren bin ich dort an einem Punkt angekommen, an dem ich mein Wissen nicht ohne weiteres erweitern konnte. Mein Modellsport Hobby hat sich inzwischen in Richtung Drohnen und Roboter entwickelt. Außerdem hatte ich bei der InoNet erste Berührungen mit der Technik welche hinter autonomen Fahrzeugen steht, was ich sehr interessant fand. So hat sich ein Studium der Geotelematik und Navigation angeboten, da dieses Studium ein breites Feld von der Elektrotechnik, über die Informatik bis hin zur, für Roboter, Drohnen und Autonome Fahrzeuge essentiellen, Lokalisierung und Navigation behandelt.

Während dem Studium hat mir vor allem die stetige praktische Anwendung und er damit verbundenen Vertiefung gefallen.

Meine vergangenen Entscheidungen bezüglich meiner Bildung und der damit verbundenen Einschränkungen zeigen wie groß meine Motivation ist neues zu lernen.

\chapter{Zu meinem Projekt}
Im Rahmen meines Projekts des Forschungsmaster, betreut von Prof. Dr. Abmayr, würde ich mich gerne mit Landmarken basierter Navigation auseinandersetzen und diese weiter erforschen.
Erste Erfahrungen mit diesem Ansatz der Navigation habe ich in meinem Praxissemester bei Objective (später Luxoft) gesammelt. Hier habe ich im Rahmen einer Zusammenarbeit mit BMW die Position anhand der aktuell gemessenen Daten und einer HD-Map erste Erfahrungen mit Landmarken basierter Navigation gesammelt.

Im Rahmen weiterer Lehrveranstaltungen in der Hochschule hatte ich die Chance die grundlegenden Algorithmen nicht nur anzuwenden sondern auch in aller Tiefe zu verstehen.

Das Projekt soll in Zusammenarbeit mit dem DLR stattfinden und auf bereits bestehenden Arbeiten aufbauen. So wurde in einer Bachelorarbeit bereits untersucht ob Straßenlaternen als Landmarken verwendet werden können. Als Sensoren wurden hier die Kameras eines Tesla verwendet. Das Ergebnis zeigt, dass dieser Ansatz Potenzial bietet aber noch weitere Untersuchung benötigt um eine zuverlässige Positionierung zu gewährleisten.

Die nötige Forschung würde ich gerne durchführen und auch gerne im Rahmen von (Forschungsmaster-) Konferenzen kommunizieren.
